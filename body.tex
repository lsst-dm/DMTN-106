\section{Introduction} \label{sec:intro}

This release procedure is general and valid for all software products in the LSST Data Management subsystem.
It is an abstraction of \citeds{SQR-016}, but it is assumed that the Jenkins builds used for the Science Pipelines can be applied in the same way to other software products. 

The release manager should be in charge of the release activities documented here.  
All developers are not affected, except for backporting of fixes into release branches, when required.

Section \ref{sec:statusAndProblems} summarizes the actual process and reports the derived problems that affect it.


\subsection{Applicable Documents}

\begin{tabbing}
AUTH-NUM\= \kill
\citeds{LDM-148} \>     DM Architecture\\
\citeds{LDM-294} \>     DM Project Management Plan   \\
\citeds{LDM-672} \>     LSST Software Release Management Policy \\
% perhaps \citell{LL:AUTH-code}\>       Software Requirements Specification for \CU,\\
\end{tabbing}


\newpage
\section{Definitions} \label{sec:definitions}

This document uses certain terms which are often overloaded. In this section, we define these terms as intended in this document.


\subsubsection{Environment} \label{sec:envdef}

An environment is a set of libraries, executables, and configurations, that are predefined for a specific context or function.

An environment is usually needed in the following contexts:

\begin{itemize}
\item software development: the environment includes all libraries and tools required to build and debug a software product
\item software test and verification: the environment includes, on top of the build and debug tools, additional tools to permit/facilitate the testing and validation activities.
\item operations: the environment shall include only tools required for the execution of an operational product. It shall not contain any build or debug tools. It shall be optimized for the operational activities. \footnote{In case a problem is found in operation the first thing to do is to replicate it in a different environment, where also debug and further analysis is possible.}
\end{itemize}

These are the most common uses of an environment. Additional scenarios can be identified.

In some cases, the same environment is used in development, tests, and operations.

Operational environment definitions should never be included in the software products.
Vanilla environment definition can be provided for unit tests purposes or as an example.


\subsection{Software Product} \label{sec:swprod}

A software product is a component of the subsystem (DM) product tree.

The following activities are done on a SW Product:
\begin{itemize}
\item development activities
\item test activities, usually unit tests and system test
\item release activities
\item packaging activities
\item operational activities, in standalone mode, in an operational pipelines part of a service. Some SW Products are libraries, therefore are not operationally, but used as dependencies by other software products.
\end{itemize}

A software product shall correspond to a single repository.
In other words, there shall be a correspondence \textit{1:1} between software products and Git repositories.
All the software product dependencies are different software products and therefore released separately.

This is not the case of DM, where a software product is comprised of multiple Git repositories.
The compromised, solution in this case, is to have a Git \textit{meta-package} that identify the software product.
All Git repositories composing the software product are dependencies in a Git \textit{meta-package} and released at the same time.
In this case, a Git package shall be related only to one software product, and it is not by himself a software product.
In other words, there is a correspondence \textit{1:many} between the software product and Git cwrepositoriespackages.
There shall never be a \textit{many:many} correspondence.

The following elements should not be part of a software product, but follow a parallel development/release process:

\begin{itemize}
\item build tools, otherwise we will have them deployed in operations.
\item environment definition (\ref{sec:envdef}) in general, with the exception of vanilla or example definitions.
\end{itemize}

Big test datasets should also not be included in a software product. 

The content of this subsection is a \textit{prerequisite} for any release procedure to be applied.


\subsection{Dependencies} \label{sec:dependencies}

As defined in the above section \ref{sec:swprod}, each software product depends on other software products.
It is not common the case where a software product has no dependencies.

The software product dependencies can be divided into two main categories:

\begin{itemize}
\item other software products developed inside the same project.
These dependencies are developed and released on their own, following the same development and release process.
Their source code is not part of the software product.
The dependency information is part of the software product, in a specific configuration file, therefore versioned altogether.
\item third party packages, developed by a team that is not part of the project.
These packages follow a completely different development and release process.
They are not part of the software product.
The dependency definition should, in general, be part of the environment definition, and therefore also not be part of the software product.
\end{itemize}

Both categories should be resolved from the binary package repository, ready to use, and not from the source code repository and rebuild before been used.
This is especially true for the third-party packages and shall be enforced when releases are done.


\subsection{Software Release} \label{sec:swrel}

A software release is a consciously identified tag of a software product repository, documented with a software release note.
The identifier is usually of the form \texttt{M.n.p}.

A software release has been proved to work before the release tag is made in the repository.
This ensures that a release meet the functionalities and fixes that are documented in the release note.

When released, a software product shall depend only from other officially released software products and third-party packages.

The tag in the Github repository and the software release note (\ref{sec:relnote}) are sufficient to identify the release, 
and therefore for a developer to resolve the dependencies, build the binaries and execute the software.

The software release is a monolithic snapshot of the software product to be used as is by the downstream processes or users.
It should be possible also to identify the release with the checksum of the release's binary package, if available.
That checksum can be used to identify which release is installed in an environment, or included in a distribution.


\subsection{Binary Packages} \label{sec:swbpkg}

A binary package is a package containing executable and/or libraries (binaries or not) for the corresponding release.
It is created by building the software provided in the release tag and creating a package with the build outcome. 
It contains also supporting scripts, documentation and other files provided but not compiled, required to run the software product.
Binary packages can be created to support multiple platforms (such as Linux, macOS, windows) if required.

Binaries packages shall be generated only once, and made available for their downstream use by dependent software or for deployment.

The majority of DM software is using eupspkg for binary packaging. Other platforms can be considered, for example \textit{conda} or \textit{pypi}.
Software products implemented in Java are using jar/war binary packaging.
%\footnote{So far there are a few technologies used to handle binary packages. 
%It is recommended to assess and optimize them, converging to use only one.
%Conda seems to have become a standard for python based projects.}


\subsection{Distribution} \label{sec:distribution}

A distribution is a collection of objects to be deployed together.
Those objects can be binary packages (ref{sec:swbpkg}), third-party packages, data, source repository, tools etc.

A distribution can be used for different purposes:

\begin{itemize}
\item to make available software releases for operations
\item to test (integration, validation, operation rehearsals) software releases
\item to provide software releases to external collaborators.
\end{itemize}

The main DM software product currently being distributed is the Science Pipeline Stack -
this may be distributed in a few ways, see \url{https://pipelines.lsst.io/install/index.html}.


\subsection{Versioning and Naming} \label{sec:versioning}
DM will  switch to Semantic Versioning, specifically Semver 2.0.0. \footnote{\url{https://semver.org/spec/v2.0.0.html}}.
This will help improve dependency management and make managing patch releases more comprehensible.
The versioning schema will have 3 digits separated by a "." . For example :
\\
$16.0.0$

Generalizing, the versioning scheme is the following:

$M.n.p$

where

\begin{itemize}
\item \textbf{M} is the major version identifier,
\item \textbf{n} is the minor version identifier,
\item \textbf{p} is the patch level identifier
\end{itemize}

The repository at this stage would be tagged $16.0.0$ .
One should note in this scheme there are no padding zeros thus:
\\
$16.1.0   \neq 16.10.0$

DM software is currently versioned using two digits separated by a "." .
For example Science Pipelines latest release:\\
$17.0.1$


\subsubsection{Branch Naming} \label{sec:branchnaming}

In order to clearly distinguish release branches from tags and other branches, the following naming scheme is suggested:

\begin{verbatim}
b.<M>.<n>.x
\end{verbatim}

where \textbf{M} is the major version and \textbf{n} is the minor version starting from 0.

The letter  \textbf{b} indicates that this is a branch.
The letter \textbf{x} is a placeholder. It indicates that versions in a specific branch will have a fixed \textbf{M.n} but different patch level, starting from \textbf{1}.


\newpage
\section{Change control} \label{sec:changecontrol}

The DMCCB is in charge of approving the changes which go in all releases (major, minor, or patch).
The DM Release Plan LDM-564 provides the expected schedule for major features, based on the P6 milestones \footnote{The DM Release Plan is in the process to be updated, see \jira{DM-17001}}.

Changes to the release plan thus are at least partially controlled the LSST change control process as milestones are moved or impacted by other changing milestones. Other changes not covered by the milestones such as features or content need to be proposed to the DMCCB using RFC Jira issues.

Patch release requests should use an RFC Jira issue which must be approved by the DMCCB.
The issue should specify what needs to be fixed in the release and why it shall specify DM issues that are requested to be included in the patch.

This process is documented in \citeds{LDM-294}, sections 3.6 and 7.4.


\subsection{Issue Management} \label{sec:issues}

The release is identified in Jira using a release issue.

All issues to be included in a release shall be added as blocking to the release issue.

Note that by introducing the field \textit{Fix in Version(s)}, will permit having the same information in a much simpler way.
Both release issue and \textit{Fix in Version(s)} fields can be used in parallel.
Once the release is done, it will also be possible to complete the field \textit{Fix in Version(s)} with the exact release,
ensuring in this way that each issue is fully documented and self consistent\footnote{process to be automated}.


\subsection{Release Note} \label{sec:relnote}

The release note is the document that identifies the SW (tag) under release in a human-readable language, easy to access and share. 
It may contain technical details and/or narrative that fully characterize the release, and that may not always be available in the software repository.

The following information should be provided:
\begin{itemize}
\item The installation instructions. This is usually a manually written set of instructions. 
\item A narrative section describing the content of the release, this summary should be written by a person not auto-generated.
The T/CAM of the product being released is responsible for making sure this is provided.
\item The list of Jira issues included in a release. This information can be extracted from Github. 
Completed epics will be highlighted while other issues will be listed to be comprehensive.
\item Technical information like the GitHub tag, dependencies, etc. can be extracted automatically from Github or other tools.
\end{itemize}

In addition, a list of implemented functionalities, requirements or completed milestones can be provided.


\newpage
\section{Software Release Procedure} \label{sec:releaseprocedure}

This release procedure has written generalizing the \textit{Stack release playbook} (\citeds{SQR-016}).

\subsection{Development} \label{sec:dev}

Development activities are not part of the release process but are the starting point for a stable and reliable master branch in all Github software packages.
The Developer Guide lays out the guidelines and process. All changes are done on ticket branches and reviewed using the Pull Request mechanism before merging to master.

Each time a change is merged into master, the following activities should be performed:

\begin{itemize}
\item continuous integration build of the Git repository (software product)
\item if unit tests pass, generate binary packages
\item build downstream dependencies: CI build on SW products that depend on the newly build SW product.
\end{itemize}

Continuous integration is done also on a ticket branch prior to merging changes to master. 

Binary packages (EUPS) and docker images for distribution are made available with the nightly and weekly builds (see next section \ref{sec:weekly}).


\subsection{Nightly and Weekly builds} \label{sec:weekly}

Nightly and weekly builds are  useful for a number of reasons such as :
\begin{itemize}
	\item Maintaining a healthy code base.
	\item Forcing us to maintains build scripts.
	\item Finding breaking changes which got checked in and passed CI.
	\item A starting point for a development activity.
	\item A starting point for a release.
\end{itemize}

Nightly builds do not generate tags in the Github repositories.
When a weekly build is done, a corresponding tag in the Github repository is created.
Though weekly builds are considered releases in Github, from a release management point of view they are not
releases - they are reference builds and, when they meet the criteria in \ref{sec:firstrc}, release candidate starting points.


\subsection{Announcement} \label{sec:anaouncement}

The preparation of the next release is announced using a community post.
This has to be done few days before the release activity starts.

In this way, all contributors will be able to provided feedback, for example additional issues expected to be included in the release.
Based on the feedback, DMCCB can take corrective actions such as delaying the release.


\subsection{First Release Candidate} \label{sec:firstrc}

The first release candidate on a release is created when:

\begin{itemize}
\item all issues that are supposed to be included in it have been implemented and merged into master.
\item a weekly build has been completed successfully.
\item the product owner confirms that the weekly is good to go. \footnote{This is not a full test, but just ensures that there are no regressions nor problems at a first glance.}
\end{itemize}

In case a weekly is presenting problems that may affect the release, the DMCCB will decide if delay the release start or kick--off the process and fix the problems in forthcoming release candidates.


Following steps have to be used to create the first release candidate:

\begin{itemize}
\item create a release candidate in GitHub (Codekit\footnote{Code management toolkit \url{https://github.com/lsst-sqre/sqre-codekit}})
\item execute a build (using Jenkins job run-rebuild)
\item create and publish the packages. For the Science Pipelines, EUPS packages are created and published in the package repository:
   \begin{itemize}
   \item source packages (using run-publish Jenkins job)
   \item binaries packages (using tarball Jenkins job)
   \end{itemize}
\item create the distribution package. Still for the Science Pipelines, a docker image is generate and made available in the docker repository:
   \begin{itemize}
   \item distribution generated and published using build-stack Jenkins job
   \end{itemize}
\end{itemize}


This procedure should be applied for all software products in the DM product tree.

The parameters required are:
\begin{itemize}
\item GIT\_REFs: the existing Git references, for example "w.2018.52". This is the stable weekly build identified.
\item GIT\_TAG: the release candidate to be generated, for example "M.n.p.rc1"
\end{itemize}

For the Science Pipelines, the above steps are collected in one single Jenkins job:

\begin{verbatim}
releases/official-release
\end{verbatim}

Note that for all software product using EUPS, like for example the Science Pipelines, release candidates need to start with a $v$.

\subsection{Other Related Artifacts}

Together with the creation of the release candidate, other artifacts need to be branched:

\begin{itemize}
\item Documentation: to consolidate all information relevant for the new release. This includes the release note.
As an example, for Science Pipelines, the package \textit{pipelines\_lsst\_io} need to be branched.
\item  Environment definition packages, like for example packages containing conda environment files (YAML).
\item  Distribution packages; still referring to the Science Pipelines, the package \textbf{lsst}  where \textit{newinstal.sh} is developed, need to be branched also.
\item Test data packages.
\item Other packages not tagged automatically, but required to be inline with the released software.
\end{itemize}


\subsection{Release Candidates Validation} \label{sec:rcvalidation}

The release candidate needs to be validated.

In an ideal case, a test campaign following a specific test plan should be conducted.
This test campaign should demonstrate that the release candidate, and therefore the forthcoming release, is behaving as required and expected.
T/CAM and product owner shall be responsible of this activity.

Practically, the validation in many cases can be:

\begin{itemize}
\item installation/configuration of the binaries packages, or distribution image; usually done manually
\item inspect of the installation, that all expected files and configuration are there
\item execution of some demo package available case by case depending of the SW product
\item try some use cases in order to prove that the release candidate is behaving properly
\item ask downstream users to act as beta testers.
\end{itemize}

For Science Pipelines, a characterization report is produced by the product owner, in order to document the release outputs.
This is a new document created for each major release.


\subsection{Resolving Problems}

In case problems are found during the validation, a DM issue needs to be created in Jira.
This issue shall follow the usual development process as described in the developer guide.
The fix will be first implemented in a ticket branch, reviewed and merged to master.
Once the fix has been proved to work, it can be backported to the release branch.

The release branch will be created only for the packages to be fixed, and will follow the naming convention described in \ref{sec:branchnaming}.

%Given a fix to backport, documented in DM-XXX Jira issue, and given a release branch bM.n.x,
Backporting mechanism can be summarized as follows \footnote{This needs to be documented in the DevGuide}:


\begin{itemize}
\item Create the backporting Jira issue DM-XYZ with subject \textit{Backport issue DM-XXX on release branch bM.n.x}
  \begin{itemize}
  \item DM-XXX Jira issue has to be already in status reviewed or done, and the corresponding ticket branch already merged to master.
  \item The ticket branch must not be deleted  until the backporting is concluded.
  \end{itemize}
\item Create the backporting ticket branch \textit{tickets/DM-XYZ}  based on DM-XXX ticket branch
\item Backport the fix, rebasing the branch \textit{tickets/DM-XYZ} from master to the release branch using the following command:
  \begin{itemize}
   \item $git rebase --onto bM.n.x$.
   \item Resolve conflicts in case there are some
  \end{itemize}
\item Open a PR from the backported ticket branch, \textit{tickets/DM-XYZ}, to merge it into the release branch
\item Merge the backported branch into the release branch when the PR is approved
\item Remove the backported ticket branch and the original ticket branch
\end{itemize}

Note that porting mechanism is a development activity, under the responsibility of the development team, and therefore needs to be documented in the developer guide and removed from this document.
This backporting procedure is just an example. Additional backporting mechanism can be proposed.

In a special case, that an issue cannot be fixed on master, the ticket branch can be opened based on the release branch.

The porting can be applied also in case that an issue, implemented on master after the the first release candidate has been created, has to be included in the release.

The DMCCB has to overview the backporting of issues to the release branch, and take corrective actions if needed. Backporting may require a considerable use of development resources and/or delay in the final release availability.


\subsection{Additional Release Candidates} \label{sec:newrc}

Once one or more issues have been fixed on the release branch, a new release candidate has to be generated. %the Jenkins job \textit{official-release} using following parameters:
The same steps used for generating the first release candidate, need to be used also in this case, with different parameters:

\begin{itemize}
\item GIT\_REFs: the starting Git references, for example "bM.n.x M.n.p.rc1". Where available the release branch will be used instead of the previous release candidate
\item GIT\_TAG: the release candidate to be generated, for example "M.n.p.rc2"
\end{itemize}

Further release candidates can be created just repeating the process with the appropriate parameters.
The release candidate 3 will require following parameters:

\begin{itemize}
\item GIT\_REFs: the starting Git references, for example "bM.n.x M.n.rc2". Where available the release branch will be used instead of the previous release candidate
\item GIT\_TAG: the release candidate to be generated, for example "M.n.rc3"
\end{itemize}

Generalizing, a release candidate N will be based on the release branch and the previous release candidate N-1.


\subsection{Final Release} \label{sec:finalrelease}

Once a final release candidate \textbf{Nf} has been identified, the final release can be created.

To create the final release, just repeat the same steps used to created the release candidates, using following parameters:

\begin{itemize}
\item GIT\_REFs: "M.n.p.rcNf"
\item GIT\_TAG: "M.n.p"
\end{itemize}

Note that the final release is from a point of view of content identical to the last release candidate.

Final tags need to be done manually for all the additional documentation, environments and other packages.

\newpage

\section{Patch Releases} \label{sec:patchreleases}

In the case that the DMCCB approves a proposed patch release, the process shall be:

\begin{itemize}
\item create the release branches from the available release tags on the Github packages impacted by the fixes (if they don't exist)
\item backport the requested issues
\item create a release candidate, for example \textit{M.n.1.rc1} using the the same approach explained above (\ref{sec:newrc})
\item validate the release candidate
\item fix eventual problems found in the release candidate and repeat the last 2 steps, create a new release candidate a validate it, until a valid release candidate is found
\item create a final release as described above (\ref{sec:finalrelease}).
\end{itemize}


