\section{Introduction} \label{sec:intro}

The scope of this document is to provide a release procedure valid for all software products in the LSST Data Management subsystem. The procedure  can of course be tailored according to specific software product needs.

\subsection{Applicable Documents}

\begin{tabbing}
AUTH-NUM\= \kill
\citeds{LDM-148} \>     DM Architecture\\
\citeds{LDM-294} \>     DM Project Management Plan   \\
%\citeds{LDM-564} \>     Data Management Releases for Verification/Integration \\
% perhaps \citell{LL:AUTH-code}\>       Software Requirements Specification for \CU,\\
\end{tabbing}


\newpage
\section{Definitions} \label{sec:definitions}

This document uses certain terms which are often overloaded -  in this section we define these terms  as intended in this document.


\subsection{Software Product} \label{sec:swprod}

A Software Product is a component of the subsystem (DM) product tree.
Releases are made of  SW products (\secref{sec:swrel}).

A SW Product should correspond to a single repository (git package).
Where a SW Product is comprised of multiple git packages
a Github \textit{metapackage} is used to identify the SW Product.
All git packages of the SW product will be dependencies in a Github \textit{metapackage} and released at the same time.

Software products \textit{live} in a software repository. The LSST software repository is Github\footnote{\url{https://github.com/lsst}}.

\subsection{Dependencies} \label{sec:dependencies}


A package may only depend on other packages. Third party dependencies are either packages or in the conda environment.

The dependency information is provided in each git package in specific dependency files  according to the build system e.g.  requirements.txt, EUPS table file, etc.
These files , and hence the dependencies,  are considered part of the source code and controlled  and versioned in the same manner.


\subsection{Software Release} \label{sec:swrel}

A software release is  a consciously identified version of a  software product
documented with a software release note.
The identifier is usually of the form NN.m (see \secref{sec:versioning}): and is used also as a  tag in the SW repository.
The tag is created on a release branch after manual checks on the last release candidate (see \secref{sec:releaseprocedure}).

The tag in the Github repository and the software release note should be sufficient for a developer
to build the binaries and execute the software.



\subsection{Binary Package} \label{sec:swbpkg}

A binary package is a package containing executable binaries for the corresponding release.
It is created by building the SW provided in the release tag.
Binary packages can be created to support multiple platforms (such as Linux, OSX, windows) if required.

Binaries in general should be generated only once, and made available for their use by dependent software or for deployment.

The majority of DM software is using EUPS for binary packaging. Other platforms can be considered, for example \textit{conda} or \textit{pypi}.
Software products implemented in java are using jar/war binary packaging.
\footnote{So far there are a few technologies used to handle binary packages. It is recommended to assess and optimize them, converging to use only one.
Conda seems to have become a standard for python based projects.}

\subsection{Distribution} \label{sec:distribution}

A distribution is a collection of binary packages to be deployed together.

A distribution can be used for different purposes:

\begin{itemize}
\item to make available software releases for operations or commissioning
\item to test (integration, validation, operation rehearsals) software releases
\item to provide software releases to external collaborators.
\end{itemize}

The main DM software product currently being distributed is the Science Pipeline Stack -
this may be used in a few ways see \url{https://pipelines.lsst.io/install/index.html}


\newpage
\subsection{Versioning and Naming} \label{sec:versioning}
DM will  switch to Semantic Versioning \footnote{\url{https://semver.org/spec/v2.0.0.html}},  this will help improve dependency management, and make  managing patch releases more comprehensible.
The versioning schema will have 3 digits separated by a "." . For example :
\\
$16.0.0$

The repository at this stage would be tagged $16.0.0$ .
One should note in this scheme there are no padding zeros thus:
\\
$16.1.0   \neq 16.10.0$

DM software is currently versioned using two digits separated by a "." .
For example Science Pipelines latest release:\\
$17.0$
The first number is the major version and the the second number is the minor version.



\subsubsection{Branch Naming}

In order to clearly distinguish release branches from tags and other branches, the following naming scheme is suggested:

\begin{verbatim}
b.<M>.<n>.x
\end{verbatim}

where {\bf M} is the major version and {\bf m} is the minor version.

The letters {\bf b} and {\bf x} indicate that this is a branch.

\newpage
\section{Change control} \label{sec:changecontrol}

The DMCCB is in charge of approving the changes which go in  all releases (major, minori, or patch).
The DM Release Plan LDM-564 provides the expected schedule for major features, based on the P6 milestones.

Changes to the release plan  thus are  at least partially controlled the LSST chancge cotrol  process as  milesotnes are moved or impacted by other changing milestones. Other changes not covered by the milestones such as features or content need to be proposed to the DMCCB using RFC Jira issues.

Patch release requests should use an RFC Jira issue which must be approved by the DMCCB.
The issue should specify what needs to be fixed in the release and why, it shall specify DM issues that are requested to be included in the patch.

This process is documented in \citeds{LDM-294}, sections 3.6 and 7.4.


\subsection{Issue Management} \label{sec:issues}

The release is identified in Jira using a release issue.

All issues to be included in a release shall be added as blocking to the release issue.

We should start to  use the field \textit{Fix in Version(s)} which is standard in Jira
but requires changes in the DM Jira project.


\newpage
\section{Release Note} \label{sec:relnote}

The release note documents the content of a release.

The following information should be  provided:
\begin{itemize}
\item Installation instructions. This is usually a manually written set of instructions.
\item Narrative section describing the content of the release, this summary should be written by an person not auto generated.
The T/CAM is responsible for making sure this is provided.
\item List of jira issues included in a release. This information can be extracted from Github. Completed epics will be highlighted while other issues will  listed to be comprehensive.
\item Technical information like the GitHub tag, dependencies,  etc. can be extracted automatically from Github or other tools.
\end{itemize}

Once the field \textit{Fix in Version(s)} has been introduced in the Jira DM project, it shall be made consistent with the release note. This process should be automated.


\newpage
\section{Software Release Procedure} \label{sec:releaseprocedure}

This release procedure has been derived from the \textit{Stack release playbook} \citeds{SQR-016}.

\subsection{Development} \label{sec:dev}

Development activities are not part of the release process, but are the starting point for a stable and reliable master branch in all Github software packages.
The \citedsp{DevGuide} developer guide lays out the guidelines and process. All changes are done on ticket branches and reviewed using the Pull Request mechanism before merging to master.

Each time a change is merged into master, the following activities should be performed:

\begin{itemize}
\item continuous integration build of the Github software package (SW product)
\item if unit tests pass, generate binary packages
\item build downstream dependencies: CI build on SW products that depend from the newly build SW product.
\end{itemize}

At the moment, continuous integration is done on a ticket branch prior to merging changes to master. Binary packages (EUPS) and docker images for distribution are made available with the nightly and weekly builds (see next section \ref{sec:weekly}).


\subsection{Daily and Weekly builds} \label{sec:weekly}

Daily and weekly builds are  useful for a number of reasons such as :
\begin{itemize}
	\item Maintaining a healthy code base.
	\item Forcing us to maintains build scripts.
	\item Finding breaking changes which got checked in and passed CI.
	\item A starting point for a development activity.
	\item A starting point for a release.
\end{itemize}

Daily builds do not generate a tag in the Github repositories.
When a weekly build is done, a corresponding tag in the Github repository is created.
Though weekly builds are considered releases in Github, from a DM release management point of view they are not
releases - they are release candidate starting points.


\subsection{Announcement} \label{sec:anaouncement}

The preparation of the next release is announced using a community post.
This has to be done few days before the release activity starts.

In this way, all contributors will be able to provided feedback, for example additional issues expected to be included in the release.
Based on the feedback, DMCCB can take corrective actions such as delaying the release.


\subsection{First Release Candidate} \label{sec:firstrc}

The first release candidate on a release is created when:

\begin{itemize}
\item all issues that are supposed to be included in it have been implemented and merged into master.
\item a weekly build has been completed successfully.
\item the product owner confirms that the weekly is good to go. \footnote{This is not a full test, but just ensure that there are no regressions nor problems at a first glance. Complete tests will be done on the release candidates.}
\end{itemize}

In case a weekly is presenting problems that may affect the release, the DMCCB will decide if delay the release start or kick--off the process and fix the problems in forthcoming release candidates.

The creation of the first release candidate is done using the Jenkins job:

\begin{verbatim}
releases/official-release
\end{verbatim}

with the following parameters:

\begin{itemize}
\item {\bf SORUCE\_GIT\_REFS}: <weekly tag> (for example \textit{w.2018.52})
\item {\bf RELEASE\_GIT\_TAG}: <Release Candidate> (for example \textit{v.17.0.rc1})
\item {\bf O\_LATEST}: false
\end{itemize}


If the Jenkins job is not available, the release candidate should be created manually:

\begin{itemize}
\item creating the release branch from the last weekly build (codekit, functionality still to be added, branches can be created manually)
\item creating a release candidate (codekit)
\item creating the binaries packages
\item creating the distribution package
\end{itemize}

As of today, the majority of these steps are integrated in Jenkins jobs. This makes it impossible to complete the release process without the use of the continuous integration system.

\subsection{Other Related Artifacts}

Together with the creation of the release candidate, other artifacts may need to be branched, and possibly have a first release candidate also:

\begin{itemize}
\item Documentation: release note are usually part of a documentation endpoint, similar to pipelines.lsst.io (for science pipelines). In order to consolidate all information relevant for the new release, a corresponding branch need to be created.
\item Environments  as needed :
	\begin{itemize}
	\item   Conda environment definition.
	\item  Envronment packageis like \textit{lsst}  where \textit{newinstal.sh} is developed for pipelines.
	\end{itemize}
\end{itemize}


\subsection{Release Candidates Validation} \label{sec:rcvalidation}

The release candidate needs to be validateid.

In an ideal case, a test campaign following a specific test plan should be conducted, demonstrating that the release candidate, and therefore the forthcoming release, is behaving as required and expected.

Practically, the validation in many cases can be:

\begin{itemize}
\item installation/configuration of the binaries packages, or distribution image; usually done manually
\item inspect of the installation, that all expected files and configuration are there
\item execution of some demo package available case by case depending of the SW product
\item try some use cases in order to prove that the release candidate is behaving properly
\item ask downstream users to act as beta testers.
\end{itemize}

For science pipelines, a characterization report is produced, in order to document the release outputs.
This is a new document created for each major release.


\subsection{Resolving Problems}

In case problems are found during the validation, a DM issue needs to be created in Jira.
This issue shall follow the usual development process as described in the developer guide.
The fix will be first implemented in a ticket branch, reviewed and merged to master.
Once the fix has been proved to work, it can be backported to the release branch.
Backporting mechanism summarized as follows \footnote{THis needs to be documented in teh DevGuide}:


{\color{red} This needs a re write}
\begin{itemize}
\item If we have an issue DM-XXX fixed on a ticket branch \textit{tickets/DM-XXX} and merged to master
the ticket branch must not be deleted  until the backporting is concluded.
\item Create a  backporting ticket DM-YYY  and a corresponding ticket branch based on DM-XXX issue
\item given a release branch created based on a release candidate, or on a weekly tag
\item backport is done from the backporting ticket branch, using the following command: \begin{verbatim}git rebase --onto <RELEASE_BRANCH>\end{verbatim} .
\begin{itemize}
\item Note that this will rebase the backporting branch to the release branch.
\end{itemize}
\item open a PR from the ported ticket branch to merge into the release branch
\item merge the ported branch into the release branch when the PR is approved
\item remove the ported branch and the original branch
\end{itemize}

Note that porting mechanism is a development activity, under the responsibility of the development team, and therefore needs to be documented in the developer guide and removed from this document.

In a special case, that an issue cannot be fixed on master, the ticket branch can be opened based on the release branch.

The porting can be applied also in case that an issue, implemented on master after the release branch has been cut, has to be included in the release.

The DMCCB has to overview the backporting of issues to the release branch, and take corrective actions if needed. Backporting may require a considerable use of development resources or delay in the final release availability.


\subsection{Additional Release Candidates} \label{sec:newrc}

Once one or more issues have been fixed on the release branch, a new release candidate has to be generated using the Jenkins job \textit{official-release} using following parameters:

\begin{itemize}
\item {\bf SOURCE\_GIT\_REFS}: <BranchId> <Previous RC>. (for example \textit{b.17.0 v17.0.rc1})
\item {\bf RELEASE\_GIT\_TAG}: <new RC> (for example \textit{v17.0.rc2})
\item {\bf O\_LATEST}: false
\end{itemize}

\subsection{Final Release} \label{sec:finalrelease}

Once a final release candidate has been identified, the final release can be created.

This has to be done using the Jenkins job \textit{official-release} as previously done for the release candidates, with the following parameters:

\begin{itemize}
\item {\bf SOURCE\_GIT\_REFS}: <Final RC>. (for example \textit{v17.0.rc2})
\item {\bf RELEASE\_GIT\_TAG}: <Release Tag> (for example \textit{17.0})
\item {\bf O\_LATEST}: true
\end{itemize}

\newpage

\section{Patch Releases} \label{sec:patchreleases}

In the case that the DMCCB approves a proposed patch release, the process shall be:

\begin{itemize}
\item create the release branches from the available release tags on the Github packages impacted by the fixes
\item backport the requested issues
\item create a release candidate, for example \textit{v17.0.1.rc1} using the the same approach explained above (\ref{sec:newrc})
\item validate the release candidate
\item fix eventual problems found in the release candidate and repeat the last 2 steps until a valid release candidate is found
\item create a final release as described above (\ref{sec:finalrelease}).
\end{itemize}


